\usepackage[utf8]{inputenc}
\usepackage{graphicx}
\usepackage[parfill]{parskip}
\usepackage[ngerman]{babel}
\usepackage[T1]{fontenc}
\usepackage[onehalfspacing]{setspace}
\usepackage{tabularx}

\usepackage{bibgerm}%Für Literaturverzeichnis

%dickere Schriftart falls gewünscht
%\usepackage{mathptmx}
%\usepackage[scaled=.90]{helvet}
%\usepackage{courier}


\usepackage{lipsum}%Zum Einfügen eines Lorem Ipsum Textes

\usepackage{enumitem} %Um Nummerieungen durch Text unterbrechen zu können
\usepackage{pdfpages}% Zum Einfügen mehrseitiger PDF-Dokumente

\usepackage[left=3cm,right=2cm,top=2cm,bottom=2cm,includehead]{geometry}%Maße für die wissenschaftliche Arbeit

\usepackage{titletoc}
%
%% Ab hier werden die Punkte für Chapter hinzugefügt, wenn gebraucht
%\titlecontents{chapter}[1.5em]{\addvspace{1pc}\normalfont\sffamily\bfseries}{\contentslabel{1.5em}}
%{\hspace*{-1.5em}}{\hspace*{0.2em} \titlerule*[0.8pc]{.}\contentspage}

\usepackage{glossaries}%Zum Erzeugen des Glossars 

%Hyperref Package für Links im Dokument
\usepackage[citecolor = black,colorlinks=true,linkcolor=black,urlcolor=black]{hyperref}

% Packete zum anzeigen von Graphen, bsp im anhang
% pgfplots
\usepackage{pgfplots}
\usepackage{pgfplotstable}
\usepgfplotslibrary{external}
\usepgfplotslibrary{dateplot}

% for pgfplots combined bar/line


%\tikzexternalize
%\tikzsetexternalprefix{images/generated/}

%\pgfplotsset{grid style={dashed,gray}}
\pgfplotsset{minor grid style={dotted,green!50!black}}
\pgfplotsset{major grid style={dashed,gray}}

\usepackage[eulergreek]{sansmath}

\tikzset{font=\sansmath\sffamily\footnotesize}
\pgfplotsset{
	%tick label style = {font=\sansmath\sffamily\footnotesize},
	%every axis label = {font=\footnotesize\sansmath\sffamily\footnotesize},
	%legend style = {font=\footnotesize\sansmath\sffamily\footnotesize},
	%label style = {font=\footnotesize\sansmath\sffamily\footnotesize},
	%compat=1.3,
	/pgf/number format/.cd,
	use comma,
	1000 sep={.}
	%every node near coord/.style={/pgf/number format/1000 sep=.},
	%every axis legend/.style={
	%y tick label style={/pgf/number format/1000 sep=.},                    
	%x tick label style={/pgf/number format/1000 sep=.},
	%}
}






