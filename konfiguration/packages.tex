\usepackage[utf8]{inputenc}
\usepackage{graphicx}
\usepackage[parfill]{parskip}
\usepackage[ngerman]{babel}
\usepackage[T1]{fontenc}
\usepackage[onehalfspacing]{setspace}
\usepackage{tabularx}

\usepackage{bibgerm}%Für Literaturverzeichnis

\usepackage{lipsum}%Zum Einfügen eines Lorem Ipsum Textes

\usepackage{enumitem} %Um Nummerieungen durch Text unterbrechen zu können
\usepackage{pdfpages}% Zum Einfügen mehrseitiger PDF-Dokumente

\usepackage[left=3cm,right=2cm,top=2cm,bottom=2cm,includehead]{geometry}%Maße für die wissenschaftliche Arbeit

\usepackage{titletoc}
%
%% Ab hier werden die Punkte für Chapter hinzugefügt, wenn gebraucht
%\titlecontents{chapter}[1.5em]{\addvspace{1pc}\normalfont\sffamily\bfseries}{\contentslabel{1.5em}}
%{\hspace*{-1.5em}}{\hspace*{0.2em} \titlerule*[0.8pc]{.}\contentspage}

\usepackage{glossaries}%Zum Erzeugen des Glossars 

%Hyperref Package für Links im Dokument
\usepackage[citecolor = black,colorlinks=true,linkcolor=black,urlcolor=black]{hyperref}







