\documentclass[a4paper,11pt,twoside,pointlessnumbers]{scrreprt}%Für einseitiges Layout Option 'twoside' entfernen
\input{konfiguration/plugin_listings} %Konfiguration für Listings
\usepackage[utf8]{inputenc}
\usepackage{graphicx}
\usepackage[parfill]{parskip}
\usepackage[ngerman]{babel}
\usepackage[T1]{fontenc}
\usepackage[onehalfspacing]{setspace}
\usepackage{tabularx}

\usepackage{cite}

\usepackage{tocbibind}%für Anzeige von Verzeichnissen im Inhaltsverzeichnis
\usepackage[hyphens]{url}%Formatierung von Links im Literaturverzeichnis

\usepackage{bibgerm}%Für Literaturverzeichnis

%dickere Schriftart falls gewünscht
\usepackage{mathptmx}
%\usepackage[scaled=.90]{helvet}
%\usepackage{courier}


\usepackage{lipsum}%Zum Einfügen eines Lorem Ipsum Textes

\usepackage{enumitem} %Um Nummerieungen durch Text unterbrechen zu können
\usepackage{pdfpages}% Zum Einfügen mehrseitiger PDF-Dokumente

\usepackage[left=3cm,right=2cm,top=2cm,bottom=2cm,includehead]{geometry}%Maße für die wissenschaftliche Arbeit

\usepackage{titletoc}
%
%% Ab hier werden die Punkte für Chapter hinzugefügt, wenn gebraucht
%\titlecontents{chapter}[1.5em]{\addvspace{1pc}\normalfont\sffamily\bfseries}{\contentslabel{1.5em}}
%{\hspace*{-1.5em}}{\hspace*{0.2em} \titlerule*[0.8pc]{.}\contentspage}

\usepackage{glossaries}%Zum Erzeugen des Glossars 



% Packete zum anzeigen von Graphen, bsp im anhang
% pgfplots
\usepackage{pgfplots}
\usepackage{pgfplotstable}
\usepgfplotslibrary{external}
\usepgfplotslibrary{dateplot}

% for pgfplots combined bar/line


%\tikzexternalize
%\tikzsetexternalprefix{images/generated/}

%\pgfplotsset{grid style={dashed,gray}}
\pgfplotsset{minor grid style={dotted,green!50!black}}
\pgfplotsset{major grid style={dashed,gray}}

\usepackage[eulergreek]{sansmath}

\tikzset{font=\sansmath\sffamily\footnotesize}
\pgfplotsset{
	%tick label style = {font=\sansmath\sffamily\footnotesize},
	%every axis label = {font=\footnotesize\sansmath\sffamily\footnotesize},
	%legend style = {font=\footnotesize\sansmath\sffamily\footnotesize},
	%label style = {font=\footnotesize\sansmath\sffamily\footnotesize},
	%compat=1.3,
	/pgf/number format/.cd,
	use comma,
	1000 sep={.}
	%every node near coord/.style={/pgf/number format/1000 sep=.},
	%every axis legend/.style={
	%y tick label style={/pgf/number format/1000 sep=.},                    
	%x tick label style={/pgf/number format/1000 sep=.},
	%}
}


%Hyperref Package für Links im Dokument
\usepackage[citecolor = black,colorlinks=true,linkcolor=black,urlcolor=black]{hyperref}



 %Hier liegen alle Packages, die im Dokument verwendet werden
\input{konfiguration/silbentrennung} %Datei, die alle Einträge zur Silbentrennung enthält
\input{konfiguration/kopfundfusszeile} %Definiert die Kopf- und Fußzeile des Dokumentes
\input{konfiguration/definitionverzeichnisse} 
\fancypagestyle{plain}{}%Setzt Kopfzeile auf allen Seiten, die plain sind!

%\input{konfiguration/glossarconfig}%Beinhaltet alles, was für ein Glossar notwendig ist inkl. Kompillierhinweise

\begin{document}
\nocite{*}

\pagestyle{empty} %Keine Anzeige von Seitenzahlen, nur der Text, der in den Dokumenten definiert ist, taucht hier auf.

 \begin{titlepage}

\begin{center}
	\includegraphics[scale=0.6]{pics/beispiellogo.png}%Uni-Logo
\end{center}
	\vspace{2cm}
      \enlargethispage{3cm}
      \begin{center}
          {\LARGE \textbf{Thema der Arbeit}}
          \vspace{3cm}
          \begin{flushleft}
              {\large \textbf{Bakkalaureusarbeit}}\\[1cm]
              zur Erlangung des Grades eines Bachelor of Science (BSc.)\\
              der Fakult\"{a}t Informatik und Elektrotechnik \\
              der Hochschule Zittau/G\"{o}rlitz -- University of Applied Sciences\\[3cm]

          vorgelegt von \\[1cm]
          {\large Name des Autors / der Autoren} \\
	

          {    \begin{tabbing}

	    
                Betrieblicher Betreuer: \= \kill
		 Matrikelnummer:  \> 12345\\
				Einrichtung: \> Hochschule Zittau/Görlitz \\
                Betreuender Professor: \> Name des Betreuers\\
                Betrieblicher Betreuer: \> Name des Betreuers\\
            \end{tabbing}}
          \vfill
              Ort, \today
          \end{flushleft}


      \end{center}
      \setcounter{page}{-1}
  \end{titlepage}
\cleardoublepage %fügt bei zweiseitigem Layout eine Leerseite ein, insofern sie benötigt wird.

\include{abstract}
\cleardoublepage

\pagestyle{plain} % Ab diesem Zeitpunkt werden die Seiten mit Kopf- und Fußzeile angezeigt
\pagenumbering{Roman} %Nummerierung der Seiten mit großen römischen Zahlen

\include{konfiguration/inhaltsvz} %Inhaltsverzeichnis


\pagestyle{empty}
\cleardoublepage

\pagestyle{fancy}
\pagenumbering{arabic} % Nummerierung arabisch

\include{content/einleitung}
\include{content/theorie}
\include{content/ist-analyse}
\include{content/loesungskonzept}
\include{content/implementierung}
\include{content/ergebnisse}
\include{content/fazit-ausblick}
\pagestyle{empty}
\cleardoublepage
\appendix % Dieser Befehl leitet den Anhang ein

\pagenumbering{alph} % Nummerierung alphanumerisch mit kleinen Buchstaben


\chapter{Anhang}
\thispagestyle{fancy}

%Dies ist ein Beispiel für einen \gls{Glossareintrag}. Bei Bedarf kann er verwendet werden.
Hier noch etwas Beispieltext.

\begin{figure}[h]
\centering
\includegraphics[scale=0.5]{pics/Beispiel.png}
\caption{Das ist ein Beispielbild.}
\end{figure}

\lipsum[1-3]

\begin{table}[h]
\centering
\begin{tabular}{|l|c|r|}
\hline
\textbf{Linksbündig} & \textbf{Zentriert} & \textbf{Rechtsbündig}\\
\hline 
Text & 123456 & $\pi$\\
\hline
\end{tabular}
\caption{Beispieltabelle}
\end{table}

\begin{figure}[htbp]
	
	\begin{tikzpicture} 
	\title{}
	\begin{axis}[ x tick label style={/pgf/number format/1000 sep=},
	%xtick=data,
	xmin=2004,
	xmax=2015,
	ymin=0,
	ymax=5,
	width=16cm,
	height=8cm,
	ylabel={Verbrauch in $MB$},
	xlabel={Zeit in Stunden $h$}
	]
	\pgfplotsset{every axis legend/.append style={
			at={(0.15,-0.1)},
			anchor=north}}
		\addplot[very thick,color=blue, mark=*] table[x=year, y=value, ignore chars=*] {
			year	value
			2013    1.40
			2012    2.80
			2011    0.20
			2010    1.60
			2009    4.90
			2008    3.50
			2007    2.70
			2006    4.20
			2005    2.20

		};
	\addlegendentry{1.Versuchsreihe - ein Datensatz}
	\end{axis} 
	\end{tikzpicture}
	\caption{Datenverbrauch pro Stunde}
\end{figure}


%Glossar einfügen, sobald es gebraucht wird

%\addcontentsline{toc}{chapter}{Glossar}{}%Fügt Kapitel zum Inhaltsverzeichnis hinzu
%\include{anhang/glossar}

\listoffigures %Abbildungsverzeichnis
\markboth{Abbildungsverzeichnis}{Abbildungsverzeichnis}


\listoftables %Tabellenverzeichnis
\markboth{Tabellenverzeichnis}{Tabellenverzeichnis}


\lstlistoflistings %Listings
\markboth{Listings}{Listings}



\include{konfiguration/abkuerzungsvz}

\bibliography{konfiguration/sources}
\bibliographystyle{apasoft}
\thispagestyle{fancy}
\markboth{Literaturverzeichnis}{Literaturverzeichnis}

\pagestyle{empty}
\cleardoublepage
\include{erklaerung} % Dieses Dokument hat keinen Seitenstil, da es als Titelseite definiert wurde. Dadurch gibt es hier auch keine Seitenzahl, die da auch nicht stehen soll.

\end{document}
