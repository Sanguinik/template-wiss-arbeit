\documentclass[a4paper,11pt,twoside,pointlessnumbers]{scrreprt}%Für einseitiges Layout Option 'twoside' entfernen
% enthält die konfiguration für das listings package
\usepackage{listings}
\usepackage{xcolor}

\lstset{language=Java}

\definecolor{lst_light_grey}
{rgb}{0.95,0.95,0.95}

\definecolor{lst_dark_grey}
{rgb}{0.8,0.8,0.8}

\definecolor{lst_highlight}
{rgb}{0,0,0.6}

\definecolor{lstgreen}
{rgb}{0,0.6,0}

\definecolor{lstmauve}
{rgb}{0.58,0,0.82}

\lstset{ %
	basicstyle=\small\ttfamily, %
	backgroundcolor=\color{lst_light_grey}, %
	captionpos=b, %
 	commentstyle=\color{lstgreen}, %
 	frame=single, %
	tabsize=2, %
%
	keywordstyle=\color{lst_highlight}, %
%
 	numbers=left, %
%
 	numberstyle=\scriptsize \color{lst_dark_grey}, %
%
 	rulecolor=\color{lst_dark_grey}, %
}

% additional highlighted keywords
\lstset { emph= {%
	var, function %
	}, emphstyle={\color{lst_highlight}}%
}
 %Konfiguration für Listings
\usepackage[utf8]{inputenc}
\usepackage{graphicx}
\usepackage[parfill]{parskip}
\usepackage[ngerman]{babel}
\usepackage[T1]{fontenc}
\usepackage[onehalfspacing]{setspace}
\usepackage{tabularx}

\usepackage{bibgerm}%Für Literaturverzeichnis

%dickere Schriftart falls gewünscht
\usepackage{mathptmx}
%\usepackage[scaled=.90]{helvet}
%\usepackage{courier}


\usepackage{lipsum}%Zum Einfügen eines Lorem Ipsum Textes

\usepackage{enumitem} %Um Nummerieungen durch Text unterbrechen zu können
\usepackage{pdfpages}% Zum Einfügen mehrseitiger PDF-Dokumente

\usepackage[left=3cm,right=2cm,top=2cm,bottom=2cm,includehead]{geometry}%Maße für die wissenschaftliche Arbeit

\usepackage{titletoc}
%
%% Ab hier werden die Punkte für Chapter hinzugefügt, wenn gebraucht
%\titlecontents{chapter}[1.5em]{\addvspace{1pc}\normalfont\sffamily\bfseries}{\contentslabel{1.5em}}
%{\hspace*{-1.5em}}{\hspace*{0.2em} \titlerule*[0.8pc]{.}\contentspage}

\usepackage{glossaries}%Zum Erzeugen des Glossars 

%Hyperref Package für Links im Dokument
\usepackage[citecolor = black,colorlinks=true,linkcolor=black,urlcolor=black]{hyperref}

% Packete zum anzeigen von Graphen, bsp im anhang
% pgfplots
\usepackage{pgfplots}
\usepackage{pgfplotstable}
\usepgfplotslibrary{external}
\usepgfplotslibrary{dateplot}

% for pgfplots combined bar/line


%\tikzexternalize
%\tikzsetexternalprefix{images/generated/}

%\pgfplotsset{grid style={dashed,gray}}
\pgfplotsset{minor grid style={dotted,green!50!black}}
\pgfplotsset{major grid style={dashed,gray}}

\usepackage[eulergreek]{sansmath}

\tikzset{font=\sansmath\sffamily\footnotesize}
\pgfplotsset{
	%tick label style = {font=\sansmath\sffamily\footnotesize},
	%every axis label = {font=\footnotesize\sansmath\sffamily\footnotesize},
	%legend style = {font=\footnotesize\sansmath\sffamily\footnotesize},
	%label style = {font=\footnotesize\sansmath\sffamily\footnotesize},
	%compat=1.3,
	/pgf/number format/.cd,
	use comma,
	1000 sep={.}
	%every node near coord/.style={/pgf/number format/1000 sep=.},
	%every axis legend/.style={
	%y tick label style={/pgf/number format/1000 sep=.},                    
	%x tick label style={/pgf/number format/1000 sep=.},
	%}
}






 %Hier liegen alle Packages, die im Dokument verwendet werden
%A
\hyphenation{ab-schi-cken}
\hyphenation{An-for-de-run-gen}
\hyphenation{an-ge-spro-che-nen}
%B
\hyphenation{be-ein-flus-sen}
\hyphenation{Bei-spiel}
\hyphenation{bei-spiels-wei-se}
\hyphenation{Be-nut-zer-o-ber-flä-che}
\hyphenation{Be-ob-ach-tung}
\hyphenation{be-reit-stellt}
\hyphenation{be-stimm-te}
%D
\hyphenation{Da-rauf da-rauf}
\hyphenation{Da-raus}
\hyphenation{Da-ten-er-he-bung}
\hyphenation{De-sig-ner}
\hyphenation{De-sign-pro-zes-ses}
\hyphenation{die-nen}
\hyphenation{dient}
%E
\hyphenation{ei-ner}
\hyphenation{Ein-ver-ständ-nis}
\hyphenation{ent-hal-ten}
\hyphenation{En-ti-ty-klas-se}
\hyphenation{ent-schei-dend}
\hyphenation{ent-wi-ckel-ten}
\hyphenation{Ent-wick-lungs-pro-zess}
\hyphenation{Ent-wick-lungs-pro-zes-ses}
\hyphenation{Ent-wick-lungs-zeit-punkt}
\hyphenation{E-va-lu-a-ti-o-nen}
\hyphenation{E-va-lu-a-tions-me-tho-de}
\hyphenation{E-va-lu-a-tions-me-tho-den}
\hyphenation{exem-pla-risch}
%F
\hyphenation{Foo-ter}
\hyphenation{freund-lich}
\hyphenation{Funk-ti-ons-um-fang}
%G
\hyphenation{ge-klärt}
\hyphenation{Ge-samt-ein-druck}
%H
\hyphenation{he-raus}
\hyphenation{hier}
%J
\hyphenation{je-weils}
\hyphenation{JPA-Im-ple-men-tie-rung}
%K
\hyphenation{kenn-zeich-nen}
\hyphenation{Klas-sen-hie-rar-chi-en}
\hyphenation{Kom-mu-ni-ka-tions-psy-cho-lo-gen}
\hyphenation{konn-te}
\hyphenation{konn-ten}
\hyphenation{könn-te}
\hyphenation{könn-ten}

%L
\hyphenation{leicht-ge-wich-ti-ge-re}
\hyphenation{Li-te-ra-tur-stu-di-um}
%M
\hyphenation{Ma-na-ged}
\hyphenation{Me-tho-de}
\hyphenation{Me-tho-den}
\hyphenation{Mensch---Sys-tem---In-ter-ak-ti-on}
\hyphenation{mit-ein-an-der}
\hyphenation{Mo-de-ra-tor}
\hyphenation{müs-sen}
%P
\hyphenation{Per-sis-tenz-lö-sung}
\hyphenation{Platt-form}
\hyphenation{Pro-be-durch-lauf}
\hyphenation{Pro-to-type}
%Q
\hyphenation{Qua-li-täts-merk-mal}
\hyphenation{Qua-li-täts-si-che-rung}
\hyphenation{Qua-li-täts-stei-ge-rung}
%R
\hyphenation{Re-fe-renz-na-me}
\hyphenation{Re-gis-trie-rung}
\hyphenation{Richt-li-ni-en}
%S
\hyphenation{Sa-xo-ni-a}
\hyphenation{Schreib-auf-wand}
\hyphenation{Schrit-te}
\hyphenation{sei-ner}
\hyphenation{setzt}
\hyphenation{si-cher-zu-stel-len}
\hyphenation{sinn-voll}
\hyphenation{Soft-ware-ent-wick-ler}
\hyphenation{Spe-zi-fi-ka-ti-on Spe-zi-fi-ka-ti-o-nen}
\hyphenation{statt-fin-den}
\hyphenation{steht}
\hyphenation{Style}
\hyphenation{Style-guide}
\hyphenation{Style-guides}
%T
\hyphenation{Teil-as-pek-ten}
\hyphenation{teil-zu-neh-men}
\hyphenation{Test-ab-läu-fe}
%U
\hyphenation{Usa-bi-li-ty}
%V
\hyphenation{ver-gleichs-wei-se}
\hyphenation{Ver-suchs-sze-na-ri-en}
\hyphenation{Vi-de-o-kon-fe-renz-an-la-ge}
\hyphenation{Voll-an-sicht}
%W
\hyphenation{Wei-ter-ent-wick-lung}
\hyphenation{We-sent-li-chen}
\hyphenation{Wire-frame}
\hyphenation{Wire-frames}
%Z
\hyphenation{Zähl-va-ri-a-blen}
\hyphenation{Zu-stand}
 %Datei, die alle Einträge zur Silbentrennung enthält
\usepackage{fancyhdr}
\pagestyle{fancy}
\fancyhf{}

%Kopfzeile links bzw. innen
\fancyhead[L]{\scshape\leftmark}
%Kopfzeile rechts bzw. außen
\fancyhead[R]{\thepage}
%Linie oben
\renewcommand{\headrulewidth}{0.5pt}

%Linie unten
%\renewcommand{\footrulewidth}{0.5pt}

\renewcommand{\chaptermark}[1]{%
\markboth{\thechapter. \ #1}{}}
 %Definiert die Kopf- und Fußzeile des Dokumentes
%Definiert, wie die Akronyme angezeigt werden sollen mit einer neuen Umgebung, die über \begin{listofacronyms} begonnen, mit \acronym{Abkürzung}{Ausgeschriebene Bedeutung} gefüllt und mit \end{listofacronyms} beendet werden kann.

%Akronyme
\newcommand{\acronym}[2]{
	\textbf{#1} \> #2\> \\
}

\newenvironment{listofacronyms}{
		%\addchap{Abkürzungsverzeichnis} %Auskommentiert, da dieser Befehl bereits in der Datei erscheint, in der die Akronyme definiert werden
  		\begin{tabbing}
		\hspace*{3cm} \= \hspace{1cm} \= \hspace{5cm} \\}{\end{tabbing}}
 
\fancypagestyle{plain}{}%Setzt Kopfzeile auf allen Seiten, die plain sind!

%Glossar

\makeglossaries
%Beispielverwendung
\newglossaryentry{Glossareintrag}{name={Glossareintrag},description={Das ist die Beschreibung des Glossareintrags und daneben steht die Seitenzahl, auf der der Eintrag zu finden ist}}

%Die Glossareinträge müssen hier definiert werden.

%Beispiel im Text: Test test \gls{Glossareintrag} test.

%zum ausführen auf konsole: pdflatex vorlage.tex danach 
% makeindex.exe -s vorlage.ist -t vorlage.glg -o vorlage.gls vorlage.glo
% danach noch mal pdflatex vorlage.tex

%END Glossar%Beinhaltet alles, was für ein Glossar notwendig ist inkl. Kompillierhinweise

\begin{document}
\nocite{*}

\pagestyle{empty} %Keine Anzeige von Seitenzahlen, nur der Text, der in den Dokumenten definiert ist, taucht hier auf.

 \begin{titlepage}

\begin{center}
	\includegraphics[scale=0.6]{pics/Beispiellogo.png}%Uni-Logo
\end{center}
	\vspace{2cm}
      \enlargethispage{3cm}
      \begin{center}
          {\LARGE \textbf{Thema der Arbeit}}
          \vspace{3cm}
          \begin{flushleft}
              {\large \textbf{Bakkalaureusarbeit}}\\[1cm]
              zur Erlangung des Grades eines Bachelor of Science (BSc.)\\
              der Fakult\"{a}t Informatik und Elektrotechnik \\
              der Hochschule Zittau/G\"{o}rlitz -- University of Applied Sciences\\[3cm]

          vorgelegt von \\[1cm]
          {\large Name des Autors / der Autoren} \\
	

          {    \begin{tabbing}

	    
                Betrieblicher Betreuer: \= \kill
		 Matrikelnummer:  \> 12345\\
				Einrichtung: \> Hochschule Zittau/Görlitz \\
                Betreuender Professor: \> Name des Betreuers\\
                Betrieblicher Betreuer: \> Name des Betreuers\\
            \end{tabbing}}
          \vfill
              Ort, \today
          \end{flushleft}


      \end{center}
      \setcounter{page}{-1}
  \end{titlepage}
\cleardoublepage %fügt bei zweiseitigem Layout eine Leerseite ein, insofern sie benötigt wird.

\thispagestyle{empty}
\section*{Kurzreferat}

Deutsches Abstract

\section*{Abstract}

Englisches Abstract

\cleardoublepage

\pagestyle{plain} % Ab diesem Zeitpunkt werden die Seiten mit Kopf- und Fußzeile angezeigt
\pagenumbering{Roman} %Nummerierung der Seiten mit großen römischen Zahlen

\include{konfiguration/inhaltsvz} %Inhaltsverzeichnis


\pagestyle{empty}
\cleardoublepage

\pagestyle{fancy}
\pagenumbering{arabic} % Nummerierung arabisch

\chapter{Einleitung}%Motivation
\thispagestyle{fancy}

Allgemeinverständliche Beschreibung des Sachkontextes und der Zielstellung - ca. 2 Seiten




\chapter{Theoretische Grundlagen}
\thispagestyle{fancy}

Die für den Untersuchungsgegenstand relevanten Themen, die über die grundlegenden Studieninhalte hinausgehen; oft auch anwendungsspezifische Aspekte - ca. 6 Seiten
\chapter{Ist-Analyse}
\thispagestyle{fancy}

Welche Defizite sollen mit der Arbeit behoben werden, welche nicht? Präzisierung der Zielstellung - ca. 6 Seiten
\chapter{Lösungskonzept}
\thispagestyle{fancy}

Wie sollen die Defizite behoben werden? Methoden, fachliche Auseinandersetzung mit alternativen Ansätzen und Auffassungen, Systembeschreibung (Architektur, Vorgehensmodell, ...) - ca. 12 Seiten


\begin{lstlisting}
public class Example {
	public static void main(String...args){
		System.out.println("hello World");
	}
}
\end{lstlisting}


\chapter{Implementierung}
\thispagestyle{fancy}

Umsetzung des Lösungskonzepts, Begründung der verwendeten Technologien - ca. 8 Seiten

Beispielcode:

\begin{lstlisting}
public class Example {
	public static void main(String...args){
		System.out.println("hello World");
	}
}
\end{lstlisting}

\chapter{Ergebnisse}
\thispagestyle{fancy}

Objektive Bewertung der vorliegenden Lösung, diverse Testverfahren, Nutzerbefragungen - ca. 4 Seiten

\chapter{Fazit und Ausblick}
\thispagestyle{fancy}

Zusammenfassung sämtlicher Ergebnisse in Bezug auf die Zielerfüllung und Vorschläge für weiterführende Arbeiten - ca. 2 Seiten


\pagestyle{empty}
\cleardoublepage
\appendix % Dieser Befehl leitet den Anhang ein

\pagenumbering{alph} % Nummerierung alphanumerisch mit kleinen Buchstaben


\chapter{Anhang}
\thispagestyle{fancy}

%Dies ist ein Beispiel für einen \gls{Glossareintrag}. Bei Bedarf kann er verwendet werden.
Hier noch etwas Beispieltext.

\begin{figure}[h]
\centering
\includegraphics[scale=0.5]{pics/Beispiel.png}
\caption{Das ist ein Beispielbild.}
\end{figure}

\lipsum[1-3]

\begin{table}[h]
\centering
\begin{tabular}{|l|c|r|}
\hline
\textbf{Linksbündig} & \textbf{Zentriert} & \textbf{Rechtsbündig}\\
\hline 
Text & 123456 & $\pi$\\
\hline
\end{tabular}
\caption{Beispieltabelle}
\end{table}

\begin{figure}[htbp]
	
	\begin{tikzpicture} 
	\title{}
	\begin{axis}[ x tick label style={/pgf/number format/1000 sep=},
	%xtick=data,
	xmin=2004,
	xmax=2015,
	ymin=0,
	ymax=5,
	width=16cm,
	height=8cm,
	ylabel={Verbrauch in $MB$},
	xlabel={Zeit in Stunden $h$}
	]
	\pgfplotsset{every axis legend/.append style={
			at={(0.15,-0.1)},
			anchor=north}}
		\addplot[very thick,color=blue, mark=*] table[x=year, y=value, ignore chars=*] {
			year	value
			2013    1.40
			2012    2.80
			2011    0.20
			2010    1.60
			2009    4.90
			2008    3.50
			2007    2.70
			2006    4.20
			2005    2.20

		};
	\addlegendentry{1.Versuchsreihe - ein Datensatz}
	\end{axis} 
	\end{tikzpicture}
	\caption{Datenverbrauch pro Stunde}
\end{figure}


%Glossar einfügen, sobald es gebraucht wird

\addcontentsline{toc}{chapter}{Glossar}{}%Fügt Kapitel zum Inhaltsverzeichnis hinzu
\printglossary[title={Glossar}, toctitle={Glossar}]

\listoffigures %Abbildungsverzeichnis
\markboth{Abbildungsverzeichnis}{Abbildungsverzeichnis}
\addcontentsline{toc}{chapter}{Abbildungsverzeichnis}{}

\listoftables %Tabellenverzeichnis
\markboth{Tabellenverzeichnis}{Tabellenverzeichnis}
\addcontentsline{toc}{chapter}{Tabellenverzeichnis}{}

\lstlistoflistings %Listings
\markboth{Listings}{Listings}
\addcontentsline{toc}{chapter}{Listings}{}


\addchap{Abkürzungsverzeichnis}%Neues Kapitel ohne Nummerierung
\markboth{Abkürzungsverzeichnis}{}%Fügt Kapitel dem Inhaltsverzeichnis und der Kopfzeile hinzu, legt zudem fest, mit welchem Namen das entsprechende Kapitel im Inhaltsverzeichnis erscheinen soll

\thispagestyle{fancy}
\begin{listofacronyms}
\acronym{GUI}{Graphical User Interface}
\end{listofacronyms}


\bibliography{konfiguration/sources}
\bibliographystyle{apasoft}
\thispagestyle{fancy}
\markboth{Literaturverzeichnis}{Literaturverzeichnis}
\addcontentsline{toc}{chapter}{Literaturverzeichnis}{}%Fügt Kapitel zum Inhaltsverzeichnis hinzu
\pagestyle{empty}
\cleardoublepage
  \begin{titlepage}
      \vspace*{5cm}
      {\underline{Eidesstattliche Erkl\"{a}rung}}\\[1cm]
      Hiermit erkl\"{a}re ich, dass ich diese Arbeit selbstst\"{a}ndig verfasst und keine anderen als die angegebenen
      Quellen und Hilfsmittel benutzt habe.\\[1cm]
      Die Arbeit wurde bisher keiner anderen Pr\"{u}fungsbeh\"{o}rde vorgelegt und auch noch nicht ver\"{o}ffentlicht.\\[4cm]
      Ort, \today
      \hfill
      \begin{tabular}{l}
          \hline
          Name des Autors / der Autoren
      \end{tabular}

      \end{titlepage}
      \if@twoside
      \newpage
      \fi % Dieses Dokument hat keinen Seitenstil, da es als Titelseite definiert wurde. Dadurch gibt es hier auch keine Seitenzahl, die da auch nicht stehen soll.

\end{document}
